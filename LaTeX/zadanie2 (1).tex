%& --translate-file=cp1250pl
\documentclass{article}
\usepackage{polski}
\usepackage{amsmath}
\usepackage{amssymb}
\usepackage{latexsym}

\author{Patryk Gradys}
\title{Matematyka}

\begin{document}
\maketitle
\tableofcontents

\section{R�wnania}
Rownanie diofantyczne
\begin{equation}\label{eq:wz1}
x^2-yz-1=0
\end{equation}
zasluguje na uwage mimo swojej prostoty. Intrygujacy jest mianowicie zwiazek
miedzy rozwiazaniami tego r�wnania w liczbach naturalnych a operacja najwiekszego
wsp�lnego dzielnika, Na fakt ten zwr�cil uwage prof. A. Schnitzel, kt�ry w
artykule pisze miedzy innymi: Nie sa znane zadne wzory zalezne od skonczenie
wielu niezaleznych parametr�w, kt�re dawalyby wszystkie rozwiazania r�wnania
(\ref{eq:wz1}) i nie zawieralaby w jakiejs formie operacji najwiekszego wsp�lnego dzielnika.
O ile zawiera dopusci sie te operacje, wystarczy przyjac

\begin{equation}
x=t_1, \indent y=t_2,t_1^2-1),  \indent z=\frac{(t_1^2-1)}{(t_2,t_1^2-1)}
\end{equation}

Aby wyznaczyc wzory rozwiazan r�wnania (\ref{eq:wz1}), czyniac zadosc wyzej postawionym
warunkom, wystarczyloby zatem zdefiniowac operacje najwiekszego
wsp�lnego dzielnika za pomoca odpowiednich dzialan na liczbach calkowitych.
W niniejszej nocie zdefiniujemy te operacje i wyznaczymy wzory rozwiazan r�wnania
(\ref{eq:wz1}) czyniace zadosc wyzej postawionym warunkom.

Niech $m, n$ beda danymi liczbami naturalnymi. W zbiorze liczb calkowitych
okreslamy funkcje $f_m^{(n)}$:

\begin{equation*}
f_m^{(n)}(x)=(x-m) \cdot (x-2m) \cdot \ldots \cdot (x-nm).
\end{equation*}

Funkcja ta ma nastepujaca wlasnosc: dla kazdej liczby naturalnej $k \le m$

\begin{equation}
f_m^{(n)}(kn)=0 \Leftrightarrow [m,n] \mid kn,
\end{equation}

gdzie $[m, n]$ jest najmniejsza wsp�lna wielokrotnoscia liczb naturalnych $m$ i $n$.
Uwaga: przez liczby naturalne rozumiemy tu liczby calkowite wieksze od zera.

Niech z kolei $F$ bedzie funkcja okreslona w zbiorze liczb calkowitych nastepujacym
wzorem

\begin{equation*}
F(x)=\frac{1}{2}+\frac{1}{2}(-1)^{2^{x^2}}.
\end{equation*}

Tak okreslona funkcja ma wlasnosc:

\begin{equation}
F(x)=\left\{
\begin{array}{ccc}
1, & \mathrm{gdy} & x \ne 0 \\
0, & \mathrm{gdy} & x = 0
\end{array}
\right.
\end{equation}

\end{document}



