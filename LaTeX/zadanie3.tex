%& --translate-file=cp1250pl
\documentclass{article}
\usepackage{polski}
\usepackage{amsmath}
\usepackage{amssymb}
\usepackage{latexsym}

\newtheorem{twierdzenie}{Twierdzenie}
\newtheorem{dowod}{DOW�D.}

\begin{document}
\section{Matematyka}
\begin{twierdzenie}[Nier�wno�� Schwarza]
Niech $z_1, z_2, \ldots, z_n \in \mathbb{C}$ oraz $\varsigma_1, \varsigma_2, \ldots, \varsigma_n \in \mathbb{C}$. W�wczas
\begin{equation*}
\left\vert
\sum^n_{j=1}z_j\overline{\varsigma_j}
\right\vert^2
\leq \sum^n_{j=1}\vert z_j \vert^2
\cdot
\sum^n_{j=1}\vert \varsigma_j \vert^2.
\end{equation*}
\end{twierdzenie}

Dow�d. Oznaczamy $A=\sum^n_{j=1} \vert z_j \vert^2$, $B=\sum^n_{j=1} \vert \varsigma_j \vert^2$, $C=\sum^n_{j=1} z_j\overline{\varsigma}_j$. Oczywiscie $A \geq 0$, $B \geq 0$.

Je�eli $B=0$, to $\varsigma_1 = \varsigma_2 = \ldots = \varsigma_n=0$ i nier�wno�� jest wtedy trywialna.

Za��my, �e $B < 0$. Mamy
\begin{align*}
0 &\leq \sum^n_{j=1} \left\vert Bz_j-C \varsigma_j \right\vert ^2 = \sum^n_{j=1}(Bz_j-C\varsigma_j)(B\overline{z_j}-\overline{C}\overline{\varsigma_j}= \\
&= \sum^n_{j=1} (B^2 \left\vert z_j \right\vert ^2 - BC \overline{z_j} \varsigma_j - B \overline{C} z_j\overline{\varsigma_j}+\left\vert C \right\vert ^2 \left\vert \varsigma_j \right\vert ^2)=\\
&= B^2 \sum^n_{j=1} \left\vert z_j \right\vert ^2 - BC \overline{\sum^n_{j=1}z_j\overline{\varsigma_j}} - B \overline{C} \sum^n_{j=1}z_j\overline{\varsigma_j} + \left\vert C \right\vert ^2 \sum^n_{j=1} \left\vert \varsigma_j \right\vert ^2 = \\
&= B^2A-B \left\vert C \right\vert ^2 - B \left\vert C \right\vert ^2 + \left\vert C \right\vert ^2 B = B^2A - B \left\vert C \right\vert ^2 = B(BA - \left\vert C \right\vert ^2).
\end{align*}
Zatem $BA - \left\vert C \right\vert ^2 \geq 0, (bo B>0), tzn. \left\vert C \right\vert ^2 \leq A \cdot B$.
\begin{twierdzenie}
a) Je�li $\alpha > 0$, to $$\lim_{n \to \infty} n^{\alpha} = +\infty, \lim_{n \to \infty} \frac{1}{n^{\alpha}} = 0$$
b) \begin{equation*}
\lim_{n \to \infty} x^n =
\left\{
\begin{array}{ccc}
0, & gdy & \vert x \vert < 1 \\
1, & gdy & x = 1 \\
+\infty, & gdy & x > 1.
\end{array}
\right.
\end{equation*}
c) Je�li $A>0$, to $\lim_{n \to \infty} \sqrt[n]{a}=1$ \\
d) $\lim_{n \to \infty} \sqrt[n]{n}=1$
\end{twierdzenie}
\begin{dowod}
a) Niech $M>0$ b�dzie dowolnie ustalone i niech $n_0=[M^{1/\alpha}]+1$, gdzie $[x]$ lub $E(x)$ oznacza najwi�ksz� liczb� ca�kowit� nie przekraczaj�c� $x$. W�wczas dla wszystkich $n$ takich, �e $n \geq n_0 > M^{1/\alpha}$ b�dziemy mieli $n^{\alpha} > M$, a wi�c $\lim_{n \to \infty}n^{\alpha}=+\infty$. Druga r�wno�� wynika z pierwszej oraz z tw. 25.

b) Przypu��my najpierw, �e $\vert x \vert < 1$. Udowodnimy, �e
\begin{equation}
\lim_{n \to \infty} x^n=0
\end{equation}
\end{dowod}
\end{document}